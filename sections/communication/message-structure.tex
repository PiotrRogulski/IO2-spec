\documentclass[../main.tex]{subfiles}
\graphicspath{{\subfix{../diagrams/}}}

\begin{document}

\subsection{Message structure}
\vspace{5mm}
This section shows how messages are structured in our system. For further information please refer to our RAML and html files (see: \textit{Attachments} section). Given examples below are represented in a JSON format.

\subsubsection{Order}
The order is represented by an object that consists of a unique id, an array of purchased items, its status and dates of creation and delivery. It also consists of clients data - their address, additional info. 
When a client makes a new order and later confirms the purchase, the database denotes the new Order. It is done through the \texttt{POST} method to \texttt{/orders/create/\{userId\}}. A \texttt{GET} request is made to \texttt{/orders/pending} when the Shop Worker is looking for all pending orders that are to be handled but when its a specific Order, \texttt{GET} is called to \texttt{/orders/\{orderId\}}. The boolean field called 'confirmedPayment' is set to \texttt{false} until the accounts are settled. The update is made by calling a \texttt{PUT} method to \texttt{/orders/\{orderId\}/payment}.

\begin{listing}[htp]
\begin{minted}{js}
{
    "orderId": "c2cf4e9e-0393-4189-af47-0aab382ce330",
    "orderItems": [
        {
            "orderItemId": "ca543631-3df7-4d06-8091-179df67c8460",
            "grossPrice": 12,
            "currency": "PLN",
            "productName": "bread",
            "quantity": 3
        }
    ],
    "creationDate": "2021-12-01 12:30:22",
    "deliveryDate": "2021-12-01 13:30:22",
    "clientAddress": {
        "street": "Prosta 12",
        "city": "Warszawa",
        "zipCode": "00-631"
    },
    "additionalInfo": "Info from the client",
    "orderStatus": "Pending",
    "confirmedPayment": false
}
\end{minted}
\caption{Message structure - Order}
\end{listing}

\subsubsection{OrderStatus}
There are several states in which the Order can be in. The following ones have been considered in the messages: Pending, InPreparation, ReadyForDelivery, PickedUpByCourier, RejectedByShop, RejectedByCustomer, Delivered. This field is modified each time some action is made upon the particular order. The methods 
\texttt{PUT} and \texttt{GET} are provided with the chosen Orders id as well as with the adequate action name. For example, by sending these requests to \texttt{/orders/\{orderId\}/pickup} one can update the status to 'PickedUpByCourier'. 

\begin{listing}[htp]
\begin{minted}{js}
    "ReadyForDelivery"
\end{minted}
\caption{Message structure - OrderStatus}
\end{listing}


\subsubsection{Client}
The clients information passed in messages are i.e.: phoneNumber and clientAddress.
In order to find more details about the client who created an Order in the \texttt{GET} method \texttt{/clients/\{clientAddress\}} is used.

\begin{listing}[htp]
\begin{minted}{js}
{
    "userId": "2bcd5428-7bb2-11ec-90d6-0242ac120003",
    "phoneNumber": "123456789",
    "clientAddress": {
        "street": "Prosta 12",
        "city": "Warszawa",
        "zipCode": "00-631"
    }
}
\end{minted}
\caption{Message structure - Client}
\end{listing}

\subsubsection{Complaint}
Each complaint can be created by a client similarly to how it's in case of an order. It has a unique id, consists of an id of an order it refers to and has a status field that can be modified by shop employees. Similarly to how the client creates a new order, the \texttt{POST} call is done to the url that ends with \texttt{/create/\{userId\}} . The \texttt{GET} method to \texttt{/complaints/pending} retrieves a list of all pending complaints.

\begin{listing}[htp]
\begin{minted}{js}
{
    "complaintId": "061ac70b-e370-40ca-a12e-9ea146ae9429",
    "orderId": "e41d4a1b-e771-4eae-84c8-c598ee60d627",
    "status": "Rejected",
    "text": "Delivery was 5 minutes late"
}
\end{minted}
\caption{Message structure - Complaint}
\end{listing}

\subsubsection{ComplaintStatus}
When the complaint was pending and is now handled by a given shop, it can either be accepted or rejected, therefore \texttt{PUT} method is called to update an appropriate record in the database. The call is made to \texttt{/complaints/\{action\}}, where \texttt{action} is one of the two mentioned possibilities.

\begin{listing}[htp]
\begin{minted}{js}
    "Rejected"
\end{minted}
\caption{Message structure - ComplaintStatus}
\end{listing}
\end{document}